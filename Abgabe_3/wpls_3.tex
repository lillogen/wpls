\documentclass[12pt,a4paper]{article}
\usepackage[ngerman]{babel}
\usepackage[T1]{fontenc}
\usepackage[utf8x]{inputenc}
\usepackage{url}
\usepackage{graphicx}
\usepackage{hyperref}
\usepackage{geometry}
\usepackage{amsfonts}
\usepackage{amsmath}
\usepackage{tabularx}
\usepackage{txfonts} %Times New Roman Font
\usepackage{titlesec} %Format der Headings ändern
\usepackage{times}
\usepackage{float}
\usepackage{listings}
\usepackage{subfig}
\usepackage[bottom]{footmisc}


\renewcommand{\thesection}{\arabic{section}.} %Nummerierung der Sections anpassen
\renewcommand{\labelenumi}{\alph{enumi})}  %Nummerierung der Listen anpassen
\titleformat{\section}{\large\bfseries}{\thesection}{0.5em}{} %Format der Section Überschrift ändern
\setlength{\parindent}{0pt} %Keine Einrückung bei neuen Paragraphen
\geometry{left=2.0cm,textwidth=17cm,top=2.5cm,textheight=23cm}

% Anpassen %
%%%%%%%%%%%%%%%%%%%%%%%%%%%%%%%%%%%%%
\newcommand{\student}{Benedict Schlüter\\ 108018212913 } % Namen eintragen
\newcommand{\partner}{Christoph Lange\\ 108015222248} % Matrikelnummer eintragen
\newcommand{\group}{1} % Gruppennummer eintragen
%%%%%%%%%%%%%%%%%%%%%%%%%%%%%%%%%%%%%

\newcommand{\hwheadtwo}{$ $
  \vspace{-2cm}
  
\noindent \student \qquad \qquad  Wireless Physical Layer Security Praktikum \hfill WS 2019/2020 \\
\noindent \partner \\
%\noindent \thirdone \\  % einkommentieren, falls ihr eine 3er Gruppe seid
\noindent Gruppe:~\group\\
$ $

  
\begin{center}    
{\Large \bf Abgabe PHYSEC 2}
\end{center}
}

\begin{document}
\hwheadtwo
Wir lassen über die Hausaufgabe eine Versionskontrolle laufen. Wenn irgendetwas in der ZIP fehlen sollte, kannst du uns entwerder 0 Punkte geben oder du schaust einmal kurz hier nach \url{https://git.noc.ruhr-uni-bochum.de/Kakashiiiiy/wpls_abgabe_2}.
\section{Messungen}
Eine schematische Zeichnung der Messumgebung ist in Abbildung \ref{fig:1_1} zu sehen.
\begin{figure}[H]
\centering
\includegraphics[width=1\textwidth]{Aufgaben_Dateien/Aufgabe_1/Zimmer.pdf} 
\caption{Zimmer aufbau}
\label{fig:1_1}
Joa die Messungen wurden durchgeführt und liegen in CSV Form in dem Ordner rum.
\end{figure}
\section{Implementierung Pearson Correlation}
Implementiert.
\section{Auswertung}
\subsection*{1}

\begin{figure}[H]
\centering
\subfloat[][]{\includegraphics[width=0.4\textwidth]{Messungen/1_ohneBwg_A_B/output_Ex3/correlation_AB.png}} \qquad
\subfloat[][]{\includegraphics[width=0.4\textwidth]{Messungen/1_ohneBwg_A_B/output_Ex3/correlation_AE.png}}
\caption{Korrelation ohne Bewegung Blocksize 100}
\label{fig:1}
\end{figure}

\subsection*{2}

\begin{figure}[H]
\centering
\subfloat[][]{\includegraphics[width=0.4\textwidth]{Messungen/2_mitBwg_A_B/output_Ex3/correlation_AB.png}} \qquad
\subfloat[][]{\includegraphics[width=0.4\textwidth]{Messungen/2_mitBwg_A_B/output_Ex3/correlation_AE.png}}
\caption{Korrelation mit Bewegung Blocksize 10}
\label{fig:2}
\end{figure}

\subsection*{3}

\subsubsection*{a)}

%OHNE BWG
\begin{figure}[H]
\centering
\subfloat[][]{\includegraphics[width=0.4\textwidth]{Messungen/3_a_ohneBwg_A_B/output_Ex3/correlation_AB.png}} \qquad
\subfloat[][]{\includegraphics[width=0.4\textwidth]{Messungen/3_a_ohneBwg_A_B/output_Ex3/correlation_AE.png}}
\caption{Korrelation ohne Bewegung \textbf{Blocksize 500}}
\label{fig:3_a_o}
\end{figure}

%MIT BWG
\begin{figure}[H]
\centering
\subfloat[][]{\includegraphics[width=0.4\textwidth]{Messungen/3_a_mitBwg_A_B/output_Ex3/correlation_AB.png}} \qquad
\subfloat[][]{\includegraphics[width=0.4\textwidth]{Messungen/3_a_mitBwg_A_B/output_Ex3/correlation_AE.png}}
\caption{Korrelation mit Bewegung \textbf{Blocksize 50}}
\label{fig:3_a_m}
\end{figure}



\subsubsection*{b)}
\begin{figure}[H]
\centering
\subfloat[][]{\includegraphics[width=0.4\textwidth]{Messungen/3_b_ohneBwg_A_B/output_Ex3/correlation_AB.png}} \qquad
\subfloat[][]{\includegraphics[width=0.4\textwidth]{Messungen/3_b_ohneBwg_A_B/output_Ex3/correlation_AE.png}}
\caption{Korrelation ohne Bewegung \textbf{Blocksize 100}}
\label{fig:3_b_o}
\end{figure}

%MIT BWG
\begin{figure}[H]
\centering
\subfloat[][]{\includegraphics[width=0.4\textwidth]{Messungen/3_b_mitBwg_A_B/output_Ex3/correlation_AB.png}} \qquad
\subfloat[][]{\includegraphics[width=0.4\textwidth]{Messungen/3_b_mitBwg_A_B/output_Ex3/correlation_AE.png}}
\caption{Korrelation mit Bewegung \textbf{Blocksize 20}}
\label{fig:3_b_m}
\end{figure}
\subsection*{4}

\begin{figure}[H]
\centering
\subfloat[][]{\includegraphics[width=0.4\textwidth]{Messungen/4_mitBwg_A_B/output_Ex3/correlation_AB.png}} \qquad
\subfloat[][]{\includegraphics[width=0.4\textwidth]{Messungen/4_mitBwg_A_B/output_Ex3/correlation_AE.png}}
\caption{Korrelation mit Bewegung \textbf{Blocksize 20}}
\label{fig:4}
\end{figure}

\subsection*{5}

\begin{figure}[H]
\centering
\subfloat[][]{\includegraphics[width=0.4\textwidth]{Messungen/5_mitBwg_A_B/output_Ex3/correlation_AB.png}} \qquad
\subfloat[][]{\includegraphics[width=0.4\textwidth]{Messungen/5_mitBwg_A_B/output_Ex3/correlation_AE.png}}
\caption{Korrelation mit Bewegung \textbf{Blocksize 10}}
\label{fig:5}
\end{figure}

\section{Quantisierer Jana Multibit}
\section{Quantisierer Mathur Suhas}
\end{document}