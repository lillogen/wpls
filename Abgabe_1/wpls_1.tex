\documentclass[12pt,a4paper]{article}
\usepackage[ngerman]{babel}
\usepackage[T1]{fontenc}
\usepackage[utf8x]{inputenc}
\usepackage{url}
\usepackage{graphicx}
\usepackage{hyperref}
\usepackage{geometry}
\usepackage{amsfonts}
\usepackage{amsmath}
\usepackage{tabularx}
\usepackage{txfonts} %Times New Roman Font
\usepackage{titlesec} %Format der Headings ändern
\usepackage{times}
\usepackage{float}
\usepackage{listings}
\usepackage[bottom]{footmisc}


\renewcommand{\thesection}{\arabic{section}.} %Nummerierung der Sections anpassen
\renewcommand{\labelenumi}{\alph{enumi})}  %Nummerierung der Listen anpassen
\titleformat{\section}{\large\bfseries}{\thesection}{0.5em}{} %Format der Section Überschrift ändern
\setlength{\parindent}{0pt} %Keine Einrückung bei neuen Paragraphen
\geometry{left=2.0cm,textwidth=17cm,top=2.5cm,textheight=23cm}

% Anpassen %
%%%%%%%%%%%%%%%%%%%%%%%%%%%%%%%%%%%%%
\newcommand{\student}{Benedict Schlüter\\ 108018212913 } % Namen eintragen
\newcommand{\partner}{Christoph Lange\\ 108015222248} % Matrikelnummer eintragen
\newcommand{\group}{1} % Gruppennummer eintragen
%%%%%%%%%%%%%%%%%%%%%%%%%%%%%%%%%%%%%

\newcommand{\hwheadtwo}{$ $
  \vspace{-2cm}
  
\noindent \student \qquad \qquad  Wireless Physical Layer Security Praktikum \hfill WS 2019/2020 \\
\noindent \partner \\
%\noindent \thirdone \\  % einkommentieren, falls ihr eine 3er Gruppe seid
\noindent Gruppe:~\group\\
$ $

  
\begin{center}    
{\Large \bf Abgabe PHYSEC 1}
\end{center}
}

\begin{document}
\hwheadtwo

\section{Spionage!}
\subsection*{a)}
Um die Nachricht zu demodulieren öffnen wir \textbf{Inspectrum} laden das Signal und wählen \textit{amplituden plot} aus. Wir passen den Kasten so an, das er unser Signal bestens erfasst. Aus diesem Plot leiten wir den \textit{derived plot} ab, der uns das demodulierte Signal anzeigt. Das demodulierte Signal findet sich im Anhang unter \textit{guess\_bin\_2.txt}
\subsection*{b)}
Nun nutzen wir das Inspectrums integriertes Tool um die Nullen und Einsen auszulesen. Dabei ist es nicht möglich das gesamte Signal auf einmal zu erfassen, da die Symbolrate leicht variiert oder die Kästen zu ungenau sind , und es zu Verzerrungen kommen kann (Oder wir waren zu ungeschickt). So war es nötig manuell das Signal 'abzutasten'. Der demodulierte Bitstring wurde mit Python umgewandelt. Die Nachricht lautet: \\\textit{UMoney4PhysecUMoney4PhysecUMoney4PhysecUMoney4PhysecUMoney4Physec} \textbf{Money4Physec}
\begin{figure}[ht]
\centering
\includegraphics[width=0.9\textwidth]{Dateien/inspectrum.png} 
\caption{Erster (fehlgeschlagener) Ansatz}
\label{fig:1}
\end{figure}
\begin{figure}[ht]
\centering
\includegraphics[width=1\textwidth]{Dateien/inspectrum_block.png} 
\caption{Zweiter Ansatz}
\label{fig:2}
\end{figure}

\newpage
\section{Funkfernbedienungen}
Generell wurde Inspectrum zum Feststellen der Symbolrate genutzt, zum dekodieren wurden Pythonscripts genutzt die irgentwo im Root der ZIP rumfliegen. Die Bilder sind unter Dateien/Funk{\_}X zu finden. In diesen Ordnern sind u.a. noch andere Datein, die wichtig für dieses Signal waren.
\subsection{Garagentoröffner}
Auf der Verpackung stand bereits die Mittelfrequenz, daher musste diese nicht ermittelt werden. Das Wasserfalldiagramm und die Zeitdarstellung sind Screenshots vom Gui-Sink Block in gnuradio. Auf die Symbolrate kommt man, indem man sich die Zeitdarstellung genauer anschaut. Dort kann man ebenfalls die Modulationstechnik perfekt erkennen. Es gibt kurze un Lange Spikes, daher wird auf ASK getippt. Zwischen den Symbolen ist die Amplitude 0, daher OOK. Den String kann man, da er so kurz ist, entwerder in der Zeitdarstellung ablesen, oder ein Teil im inneren der Funkfernbedienung wie in Abb. \ref{fig:6} zu sehen
\begin{table}[H]
\centering
\begin{tabular}{|l|l|}
\hline
Name & Nice Easy-Transmitter \\
\hline
Center Frequency& 27,12Mhz \\
\hline
Wasserfalldiagramm & Abb. \ref{fig:3} \\
\hline
Zeitdarstellung & Abb. \ref{fig:4} \\
\hline
Symbolrate & 2.6kBd\\
\hline
Modulationstechnik & ASK-OOK Abb. \ref{fig:4} \\
\hline
Demodulierter String & 0000010101010 Abb. \ref{fig:5} und \ref{fig:6}\\
\hline
verschiedene Aktionen & Das Ding hat halt nur einen Button \\
\hline
\end{tabular}
\end{table}
\begin{figure}[H]
\centering
\includegraphics[width=1\textwidth]{Dateien/Funk_1/Garage_Waterfall.png} 
\caption{Wasserfall Diagramm}
\label{fig:3}
\end{figure}
\begin{figure}[H]
\centering
\includegraphics[width=0.5\textwidth]{Dateien/Funk_1/Garagen_foto.jpg} 
\caption{Foto des Garagenöffners von innen}
\label{fig:6}
\end{figure}
\begin{figure}[H]
\centering
\includegraphics[width=1\textwidth]{Dateien/Funk_1/Garage_Constellation.png} 
\caption{Zeitdarstellung}
\label{fig:4}
\end{figure}
\begin{figure}[H]
\centering
\includegraphics[width=1\textwidth]{Dateien/Funk_1/Garage_Demod.png}
\caption{Demoduliertes Signal(kurz steht für 0 Lang für 1)} 
\label{fig:5}
\end{figure}
Zeitachsenbeschriftung ist in \ref{fig:5} falsch, da die falsche Samplerate im Time-Sink eingestellt wurde. Ich glaube es wurde mit 50 dezimiert, also man müsste die Achse mal 50 rechnen.
\subsection{Autoschlüssel AM}
\begin{table}[H]
\begin{tabular}{|l|l|}
\hline
Name & Autoschlüssel VW-Golf-7 \ref{fig:schlu} \\
\hline
Center Frequency& 434,42Mhz \\
\hline
Wasserfalldiagramm & Abb. \ref{fig:7} \\
\hline
Zeitdarstellung & Abb. \ref{fig:8} \\
\hline
Symbolrate & 3.2kBd  \\
\hline
Modulationstechnik & ASK Abb. \ref{fig:11} ; meiste Energie in der Mitte, bei FSK verteilter \\
\hline
Demodulierter String &   Abb. \ref{fig:12} und \ref{fig:13}\\
\hline
verschiedene Aktionen & Auf; Zu ; Kofferraum, ja natürlich ist der Bitstring hier unterschiedlich \ref{fig:13}\\
\hline
\end{tabular}
\end{table}

\begin{figure}[H]
\centering
\includegraphics[width=0.5\textwidth]{Dateien/Funk_2/autoschluessel.jpg}
\caption{Technische Daten des Schlüssels} 
\label{fig:schlu}
\end{figure}
\begin{figure}[H]
\centering
\includegraphics[width=0.5\textwidth]{Dateien/Funk_2/Signal_gqrx.png}
\caption{Wasserfall Diagramm} 
\label{fig:7}
\end{figure}
\begin{figure}[H]
\centering
\includegraphics[width=1\textwidth]{Dateien/Funk_2/complete_1.png}
\caption{Zeitdarstellung des Signals/Signal geht über mehr als eine halbe Sekunde} 
\label{fig:8}
\end{figure}
\begin{figure}[H]
\centering
\includegraphics[width=0.8\textwidth]{Dateien/Funk_2/one_of_3.png}
\caption{In den letzen der 3 Blöcke reingezoomt, sieht sehr homogen bis auf das letzte Drittel aus.} 
\label{fig:10}
\end{figure}
\begin{figure}[H]
\centering
\includegraphics[width=1\textwidth]{Dateien/Funk_2/guessed_mod.png}
\caption{Auf Grundlage dieser Darstellung auf ASK getippt; weiter in Abb. \ref{fig:10} gezoomt} 
\label{fig:11}
\end{figure}
\begin{figure}[H]
\centering
\includegraphics[width=0.8\textwidth]{Dateien/Funk_2/4_bit.png}
\caption{Ich denke es werden 4 Symbole übertragen (Kurz aus/an, lang an/aus)} 
\label{fig:12}
\end{figure}
Anfangs hatten wir große Probleme an die demodulierten Bits zu kommen, doch nach gründlicher Lektüre einiger Seiten des GNU-Radio Wikis, war es uns möglich die Werte in eine Datei zu schreiben, wie wir es haben wollten (binär). Nun mussten wir nur noch die Symbole zuordnenn und konnten decodieren. Die These, das 4 Symbole genutzt wurden, erwies sich als falsch. Nach diesem Schema decodiert, sah der String viel zu homogen aus, dafür, dass er einen kyptographischen Rolling-Code nutzt.
\begin{figure}[H]
\centering
\includegraphics[width=1\textwidth]{Dateien/Funk_2/3_demod_string.png}
\caption{3 Bitstrings (OOK-2Bit) von oben nach unten: schließen, öffnen, Kofferraum}
\label{fig:13}
\end{figure}
\begin{figure}[H]
\centering
\includegraphics[width=1\textwidth]{Dateien/Funk_2/gnu_radio_aufbau.png}
\caption{GNU Radio Blocks} 
\label{fig:9}.
\end{figure}
\subsection*{Autoschlüssel FM}
Dieser Autoschlüssel verwerndet FM, um die Sequenz zu dekodieren brauchen wir also ein anderes Python Script bzw einen anderen GNU-radio-Graphen. Da Inspectrum eine niedrige Frequenz als 0 interpretiert und eine hohe als 1, demodulieren wir den String wieder mit einem Python Script.
\begin{table}[H]
\begin{tabular}{|l|l|}
\hline
Name & Autoschlüssel Smart \ref{fig:schlu} \\
\hline
Center Frequency& 27,52Mhz \\
\hline
Wasserfalldiagramm & Abb. \ref{fig:14} \\
\hline
Zeitdarstellung & Abb. \ref{fig:15} \\
\hline
Symbolrate & 4kBd  \\
\hline
Modulationstechnik & FSK Abb. \ref{fig:15}\\
\hline
Demodulierter String &   Abb. \ref{fig:16} der obere öffnet\\
\hline
verschiedene Aktionen & Auf; Zu ; Kofferraum, Bitstrings sind unterschiedlich \ref{fig:16}\\
\hline
\end{tabular}
\end{table}
\begin{figure}[H]
\centering
\includegraphics[width=1\textwidth]{Dateien/Funk_3/wasserfall.png}
\caption{Wasserfalldiagramm} 
\label{fig:14}.
\end{figure}
\begin{figure}[H]
\centering
\includegraphics[width=1\textwidth]{Dateien/Funk_3/fm_mod.png}
\caption{Zeitdarstellung des Signals} 
\label{fig:15}.
\end{figure}
\begin{figure}[H]
\centering
\includegraphics[width=1\textwidth]{Dateien/Funk_3/auf_zu_kofferraum.png}
\caption{Auf, Zu, Kofferraum Signal} 
\label{fig:16}.
\end{figure}
\section{Theorie Link Budget}
\subsection*{a)}
\begin{eqnarray*}
\log_{10}{4} \approx 6 \\
20*\log_{10}{(\frac{120*10^3*1*10^9*4*\pi}{299792.458}}) \approx 134 \\
6-4+19-134+6+36 = -71
\end{eqnarray*}
Benes Matrikelnummer endet auf 13, also müsste das Signal mit $13+71=84$db 'erzeugt' werden
\subsection*{b)}
\begin{eqnarray*}
100+6-4+19-x+6+36 = 50 \\
x = 103 \\
20*\log_{10}{(\frac{120*10^3*freq*4*\pi}{299792.458}}) = 103 \\
\frac{120*10^3*freq*4*\pi}{299792.458} = 10^{\frac{103}{20}} \\
freq = \frac{10^{\frac{103}{20}} * 299792.458}{120*10^6*4*\pi} \\
freq = 28082\\
\end{eqnarray*}
Die Frequenz müsste ca 28.1KHz betragen.
\subsection*{c)}
\begin{eqnarray*}
\frac{dist*1*10^9*4*\pi}{299792.458} = 10^{\frac{103}{20}} \\
dist = \frac{10^{\frac{103}{20}} * 299792.458}{10^9*4*\pi} \\
dist = 3.369\\
\end{eqnarray*}
Die Distanz müsste 3.37 Meter betragen.
\subsection*{d)}
\begin{eqnarray*}
10^{14.3} = 199526231496888 \\
10^{14} = 100000000000000 \\
2^{47} = 140737488355328 \\
10^{14}*2 = 200000000000000 \\
\end{eqnarray*}
Ja, es wird in diesem Fall nicht besser, wenn man beide Methoden kombiniert.
\section{Reading Assignment}
\subsection*{a)}
Durch DSSS und FHSS wird das Signal auf eine größere Bandbreite verteilt (DSSS: ganze Signal im F-Spektrum verteilt und geschwächt; FHSS: Bursts auf unterschiedlichen Frequenzen). Dadurch müsste der Angreifer insgesamt mit mehr Leitung senden, da er einen größeren Frequenzbereich jammen müsste.
\subsection*{b)}
Die Industriellen Standarts für WMAN sind WiMAX sowie LTE. Die Übertragungsrate reicht bis zu 1Gb/s und hat eine Übertragungsreichweite von bis zu 100km.
\end{document}
