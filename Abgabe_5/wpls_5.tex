\documentclass[12pt,a4paper]{article}
\usepackage[ngerman]{babel}
\usepackage[T1]{fontenc}
\usepackage[utf8x]{inputenc}
\usepackage{url}
\usepackage{graphicx}
\usepackage{hyperref}
\usepackage{geometry}
\usepackage{amsfonts}
\usepackage{amsmath}
\usepackage{tabularx}
\usepackage{txfonts} %Times New Roman Font
\usepackage{titlesec} %Format der Headings ändern
\usepackage{times}
\usepackage{float}
\usepackage{listings}
\usepackage{subfig}
\usepackage[linesnumbered, lined, boxed, commentsnumbered]{algorithm2e}
\usepackage[bottom]{footmisc}


\renewcommand{\thesection}{\arabic{section}.} %Nummerierung der Sections anpassen
\renewcommand{\labelenumi}{\alph{enumi})}  %Nummerierung der Listen anpassen
\titleformat{\section}{\large\bfseries}{\thesection}{0.5em}{} %Format der Section Überschrift ändern
\setlength{\parindent}{0pt} %Keine Einrückung bei neuen Paragraphen
\geometry{left=2.0cm,textwidth=17cm,top=2.5cm,textheight=23cm}

% Anpassen %
%%%%%%%%%%%%%%%%%%%%%%%%%%%%%%%%%%%%%
\newcommand{\student}{Benedict Schlüter\\ 108018212913 } % Namen eintragen
\newcommand{\partner}{Christoph Lange\\ 108015222248} % Matrikelnummer eintragen
\newcommand{\group}{1} % Gruppennummer eintragen
%%%%%%%%%%%%%%%%%%%%%%%%%%%%%%%%%%%%%

\newcommand{\hwheadtwo}{$ $
  \vspace{-2cm}
  
\noindent \student \qquad \qquad  Wireless Physical Layer Security Praktikum \hfill WS 2019/2020 \\
\noindent \partner \\
%\noindent \thirdone \\  % einkommentieren, falls ihr eine 3er Gruppe seid
\noindent Gruppe:~\group\\
$ $

  
\begin{center}    
{\Large \bf Abgabe PHYSEC 4}
\end{center}
}

\begin{document}
\hwheadtwo
Wir lassen über die Hausaufgabe eine Versionskontrolle laufen\url{https://github.com/lillogen/wpls}.\\

\section{Implementierung, Analyse, Theorie}
\subsection*{Analyse 1}
- Das obrige Messsetup macht es möglich genauere Messwerte zu generieren, da die Umwelt einen größeren Einfluss nimmt. Außerdem ist es möglich, durch komplexe Phasenwerte, deutlich mehr Bits zu extrahieren. Das sorgt für mehr Sicherheit gegen Angreifer, da die Angriffe komplexer werden.\\\\
- Die Wahrscheinlichkeit, dass false-positive Werte auftreten, ist bei RSSI-Werten geringer. Das liegt daran, dass diese eben nicht so empfindlich reagieren. Dadurch kann kann ein gemeinsamer Schlüssel einfacher extrahiert werden. 
\subsection*{Theorie}
- Intradistanzen sind die Abstände zweier PUF-responses, die mit der gleichen Challenge erzeugt wurden.\\\\
- Interdistanzen sind die Abstände zweier PUF-responses, die mit unterschiedlichen Challenges erzeugt werden.
\subsection*{Implementierung}
Die Implementierung ist in der Datei...
\subsection*{Analyse 2}
Eine Überlappung zwischen Intra und Interdistanzen ist schlecht, das heißt, dass wir bei einer Messung nicht entscheiden können, ob es nur eine normale Abweichung ist, oder ob die Spiegelverstellt wurden.
Je geringer die Überlappung ist, desto stärker ist die PUF. 
Ohne Preprocessing hat es ein Angreifer einfach (hohe Überlappung), aber sobald wir Preprocessing nutzen, hat der Angreifer keine Chance mehr, da Inter und Intra Distanzen sich nicht mehr überlappen.
\newpage

\section{Reading Assignment}
- Die effektive Bandbreite beim Indoor Positioning muss 360 MHz betragen, um 1 cm genaue Positionen bestimmen zu können. Das Problem mit handelsüblichen WlAN AP's besteht darin, dass deren Brandbreite nur 20-40MHz betragen, was für eine genau Positionsbestimmung indoor nicht ausreicht.\\\\
- Um Wireless Event Detection zu ermöglichen müssen genau zwei Phasen durchlaufen werden. Zum ersten ein Offline Training, welches bei jedem Indoor Event, aus den entsprechenden CSI Daten eine Matrix erstellt. Und zum zweiten das Online testing, welches das Eintreten, von Indoor Events durch Äbschätzung von Ähnlichkeit zwischen Test und Trainings CSI zu erkennen.\\\\
- Die Human Radio Biometrics werden durch die physikalischen Eigenschaften Größe, Masse, Wasseranteil oder sogar Haut beeinflusst. Außerdem ist der Mensch nicht reflektiv und zusätzlich permissiv. Dies verhindert, dass Human Radio Biometrics gute Messwerte ergeben. \\\\
- Es gibt zwei Arten von Indoor Tracking. Zum einen Indoor Tracking triangulation based. Dabei wird die Entfernung mit dem zugehörigen Winkel zwischen Gerät und Ankern abgeschätzt, um die geometrische Triangulation auf die Lage anzuwenden. So lässt sich die Lage des Geräts festlegen. Des Weiteren gibt es Indoor Tracking fingerprinting based. Dafür müssen fingerprints der Lage generiert und regelmäßig erneuert werden. Mit diesen fingerprints, gespeichert in einer Datenbank, ist es dann auch möglich, den Standort eines Geräts ausfindig zu machen.
\end{document}