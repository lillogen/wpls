\documentclass[12pt,a4paper]{article}
\usepackage[ngerman]{babel}
\usepackage[T1]{fontenc}
\usepackage[utf8x]{inputenc}
\usepackage{url}
\usepackage{graphicx}
\usepackage{hyperref}
\usepackage{geometry}
\usepackage{amsfonts}
\usepackage{amsmath}
\usepackage{tabularx}
\usepackage{txfonts} %Times New Roman Font
\usepackage{titlesec} %Format der Headings ändern
\usepackage{times}
\usepackage{float}
\usepackage{listings}
\usepackage{subfig}
\usepackage[linesnumbered, lined, boxed, commentsnumbered]{algorithm2e}
\usepackage[bottom]{footmisc}


\renewcommand{\thesection}{\arabic{section}.} %Nummerierung der Sections anpassen
\renewcommand{\labelenumi}{\alph{enumi})}  %Nummerierung der Listen anpassen
\titleformat{\section}{\large\bfseries}{\thesection}{0.5em}{} %Format der Section Überschrift ändern
\setlength{\parindent}{0pt} %Keine Einrückung bei neuen Paragraphen
\geometry{left=2.0cm,textwidth=17cm,top=2.5cm,textheight=23cm}

% Anpassen %
%%%%%%%%%%%%%%%%%%%%%%%%%%%%%%%%%%%%%
\newcommand{\student}{Benedict Schlüter\\ 108018212913 } % Namen eintragen
\newcommand{\partner}{Christoph Lange\\ 108015222248} % Matrikelnummer eintragen
\newcommand{\group}{1} % Gruppennummer eintragen
%%%%%%%%%%%%%%%%%%%%%%%%%%%%%%%%%%%%%

\newcommand{\hwheadtwo}{$ $
  \vspace{-2cm}
  
\noindent \student \qquad \qquad  Wireless Physical Layer Security Praktikum \hfill WS 2019/2020 \\
\noindent \partner \\
%\noindent \thirdone \\  % einkommentieren, falls ihr eine 3er Gruppe seid
\noindent Gruppe:~\group\\
$ $

  
\begin{center}    
{\Large \bf Abgabe PHYSEC 2}
\end{center}
}

\begin{document}
\hwheadtwo
Wir lassen über die Hausaufgabe eine Versionskontrolle laufen. Wenn irgendetwas in der ZIP fehlen sollte, kannst du uns entwerder 0 Punkte geben oder du schaust einmal kurz hier nach \url{https://git.noc.ruhr-uni-bochum.de/Kakashiiiiy/wpls_abgabe_3}.
\section{Messungen}
Eine schematische Zeichnung der Messumgebung ist in Abbildung \ref{fig:1_1} zu sehen.
\begin{figure}[H]
\centering
\includegraphics[width=1\textwidth]{Aufgaben_Dateien/Aufgabe_1/Zimmer.pdf} 
\caption{Zimmer aufbau}
\label{fig:1_1}
Die Messungen wurden durchgeführt und liegen in CSV Form in dem Ordner rum.
\end{figure}
\section{Implementierung Pearson Correlation}
Implementiert.
\section{Auswertung}
Wir haben zwei Ziele, zum einen wollen wir, dass Eve und Bob möglichst unterschiedliche Impulsantworten zu Alice haben, zum anderen, dass wir möglichst viel Entropie auf dem Channel haben, um möglichst viele Schlüsselbits ableiten zu können.\\
Die Bilder mit blauen Punkten sind die Korrelation A-B mit B-A und die mit den roten Punkten B-A mit A-E.
\subsection*{1}
In der Abbildung \ref{fig:1} ist zu sehen, dass die Werte doch recht unkorreliert sind. Sowohl von Alice und Bob, als auch von Eve. Dadurch, dass wir uns nicht bewegen ist wenig Entropie im Raum, es verändert sich die Umgebung also nicht. Es hat sich nichts geändert, daher war der RSSI-Wert relativ konstant und die Standardabweichung sehr sehr gering. Um so einer Division durch 0 zu entgehen haben wir die 0-Werte rausgenommen. Die Werte sind so unkorreliert, da der Channel unidirektional durch Interferenz verändert wird und auch thermal noise hier eine größere Rolle spielt. 
\begin{figure}[H]
\centering
\subfloat[][]{\includegraphics[width=0.4\textwidth]{Messungen/1_ohneBwg_A_B/output_Ex3/correlation_AB.png}} \qquad
\subfloat[][]{\includegraphics[width=0.4\textwidth]{Messungen/1_ohneBwg_A_B/output_Ex3/correlation_AE.png}}
\caption{Korrelation ohne Bewegung Blocksize 10}
\label{fig:1}
\end{figure}
\begin{table}[H]
\centering
\begin{tabular}{l|l|l}
& Mittelwert & empirische Standardabweichung \\
\hline
Alice & -68.48307 & 0.69923 \\
\hline
Bob & -68.30611 & 0.66904 \\
\hline
Eve & -54.80578 & 0.41131 \\
\end{tabular}
\caption{absolute Daten der 1. Messung}
\end{table}
\subsection*{2}
In Abbildung \ref{fig:2} sieht es schon ganz anders aus. Bei dieser Messung haben wir uns im Raum bewegt. Hier sieht man deutlich, das die RSSI Werte von Alice und Bob sehr korreliert sind, die von Alice und Eve hingegen überhaupt nicht. Dieses Ergebnis haben wir erwartet, es ist wie in der Vorlesung beschrieben eingetreten.
\begin{figure}[H]
\centering
\subfloat[][]{\includegraphics[width=0.4\textwidth]{Messungen/2_mitBwg_A_B/output_Ex3/correlation_AB.png}} \qquad
\subfloat[][]{\includegraphics[width=0.4\textwidth]{Messungen/2_mitBwg_A_B/output_Ex3/correlation_AE.png}}
\caption{Korrelation mit Bewegung Blocksize 10}
\label{fig:2}
\end{figure}
\begin{table}[H]
\centering
\begin{tabular}{l|l|l}
& Mittelwert & empirische Standardabweichung \\
\hline
Alice & -70.24570 & 5.23857 \\
\hline
Bob & -70.02834 & 5.12334 \\
\hline
Eve & -58.23386 & 4.41267 \\
\end{tabular}
\caption{absolute Daten der 2. Messung}
\end{table}
\subsection*{3}
\subsubsection*{a) ohne Bewegung}
Wie man sieht, sieht man nichts. Durch eine kurze Entfernung wird der Wert relativ konstant gehalten, Reflektionen etc. sind im Vergleich zum Ursprungssignal so schwach, dass sie kaum Auswirkung haben, daher haben wir auch hier ein RSSI Wert, der die ganze Zeit konstant war. Dies hat zur Folge, dass die Varianz 0 ist, da wir durch 0 teilen, existiert also keine Korrelation. Eve's Wert war auch relativ konstant aber weicht dennoch leicht von Bob's ab.
%OHNE BWG
\begin{figure}[H]
\centering
\subfloat[][]{\includegraphics[width=0.4\textwidth]{Messungen/3_a_ohneBwg_A_B/output_Ex3/correlation_AB.png}} \qquad
\subfloat[][]{\includegraphics[width=0.4\textwidth]{Messungen/3_a_ohneBwg_A_B/output_Ex3/correlation_AE.png}}
\caption{Korrelation ohne Bewegung \textbf{Blocksize 10}}
\label{fig:3_a_o}
\end{figure}
\begin{table}[H]
\centering
\begin{tabular}{l|l|l}
& Mittelwert & empirische Standardabweichung \\
\hline
Alice & -49.00072 & 0.02689 \\
\hline
Bob & -49.00000 & 0.00000 \\
\hline
Eve & -72.23263 & 0.85618 \\
\hline
\end{tabular}
\caption{absolute Daten der 3 Messung ohne Bewegung}
\end{table}

\subsubsection*{a) mit Bewegung}
Bringen wir nun Bewegung zwischen Alice und Bob sieht die Sache schon wieder ganz anders aus. Da wir teilweise mit der Hand zwischen A und B waren, haben wir dort eine komplett andere Impulsantwort als Eve 'empfangen'. Dies hat zu Folge, das wir sehr korrelierte Werte zwischen Alice und Bob haben aber nicht zwischen Alice und Eve. Die Einrüche kommen durch Bewegungspausen.
%MIT BWG
\begin{figure}[H]
\centering
\subfloat[][]{\includegraphics[width=0.4\textwidth]{Messungen/3_a_mitBwg_A_B/output_Ex3/correlation_AB.png}} \qquad
\subfloat[][]{\includegraphics[width=0.4\textwidth]{Messungen/3_a_mitBwg_A_B/output_Ex3/correlation_AE.png}}
\caption{Korrelation mit Bewegung \textbf{Blocksize 10}}
\label{fig:3_a_m}
\end{figure}

\begin{table}[H]
\centering
\begin{tabular}{l|l|l}
& Mittelwert & empirische Standardabweichung \\
\hline
Alice & -48.80779 & 1.79983 \\
\hline
Bob & -48.79270 & 1.79899 \\
\hline
Eve & -69.58161 & 3.10826 \\
\end{tabular}
\caption{absolute Daten der 2. Messung mit Bewegung}
\end{table}

\subsubsection*{b) ohne Bewegung}
Durch die hohe Entfernung wirken sich nun schon kleinste Veränderungen auf den RSSI Wert aus, die absoluten Werte liegen um die -85 dB. Es gab keine direkte Verbindung mehr. Es kamen also nur noch reflektierte Signale an. Die Werte sind so unkorreliert, da Signale teilweise verlorengegangen sind oder zu schwach waren und nicht angekommen sind. Dies geschah aber nur in eine Richtung. Dass die Werte so unterschiedlich sind liegt daran, dass die Hardware nicht perfekt ist. Das Stichwort lautet 'thermal noise' und es kann so zu kleinen Fehler kommen. Des Weiteren gibt es zufällige Veränderungen des Channel, die nicht bidirektional sind, das heißt, nur in eine Richtung. Dies kommt aufgrund der hohen Abständen mehr zum Vorschein als bei anderen Messungen. 

\begin{figure}[H]
\centering
\subfloat[][]{\includegraphics[width=0.4\textwidth]{Messungen/3_b_ohneBwg_A_B/output_Ex3/correlation_AB.png}} \qquad
\subfloat[][]{\includegraphics[width=0.4\textwidth]{Messungen/3_b_ohneBwg_A_B/output_Ex3/correlation_AE.png}}
\caption{Korrelation ohne Bewegung \textbf{Blocksize 10}}
\label{fig:3_b_o}
\end{figure}

\begin{table}[H]
\centering
\begin{tabular}{l|l|l}
& Mittelwert & empirische Standardabweichung \\
\hline
Alice & -85.61755 & 0.61814 \\
\hline
Bob & -84.66771 & 0.60191 \\
\hline
Eve & -87.68534 & 1.21669 \\
\end{tabular}
\caption{absolute Daten der 3. Messung ohne Bewegung}
\end{table}

\subsubsection*{b) mit Bewegung}
Mit Bewegungen sehen wir, dass wir wieder eine sehr starke Korrelation zwischen Alice und Bob haben, aber keine zwischen Alice und Eve. Die absolute Wertespanne, in der sich die RSSI-Werte befinden hat sich im Vergleich zu dem Versuch ohne Bewegung vergrößert. So sind die absoluten Werte zwischen -79 und -100. Wir haben ein ähnliches Bild, wie bei den vorangegangenen Aufgaben mit Bewegung. 
%MIT BWG
\begin{figure}[H]
\centering
\subfloat[][]{\includegraphics[width=0.4\textwidth]{Messungen/3_b_mitBwg_A_B/output_Ex3/correlation_AB.png}} \qquad
\subfloat[][]{\includegraphics[width=0.4\textwidth]{Messungen/3_b_mitBwg_A_B/output_Ex3/correlation_AE.png}}
\caption{Korrelation mit Bewegung \textbf{Blocksize 10}}
\label{fig:3_b_m}
\end{figure}

\begin{table}[H]
\centering
\begin{tabular}{l|l|l}
& Mittelwert & empirische Standardabweichung \\
\hline
Alice & -86.38783 & 2.98428 \\
\hline
Bob & -85.40540 & 3.13647 \\
\hline
Eve & -87.00185 & 4.72047 \\
\end{tabular}
\caption{absolute Daten der 3. Messung mit Bewegung}
\end{table}

\subsection*{4}
Durch die Bewegung des Knoten B erhöhen wir die Entropie immenz. Wie in Abbildung \ref{fig:4} zu sehen erzielen wir unser bis hierhin bestes Ergebnis. Eve hat überhaupt keine Chance an den Schlüssel zu kommen, da die Werte extrem unkorreliert im Vergleich zu Bob-Alice sind. Die Korrelation ist sehr oft genau 1 und schankt nur am Anfang ein wenig
\begin{figure}[H]
\centering
\subfloat[][]{\includegraphics[width=0.4\textwidth]{Messungen/4_mitBwg_A_B/output_Ex3/correlation_AB.png}} \qquad
\subfloat[][]{\includegraphics[width=0.4\textwidth]{Messungen/4_mitBwg_A_B/output_Ex3/correlation_AE.png}}
\caption{Korrelation mit Bewegung \textbf{Blocksize 10}}
\label{fig:4}
\end{figure}

\begin{table}[H]
\centering
\begin{tabular}{l|l|l}
& Mittelwert & empirische Standardabweichung \\
\hline
Alice & -62.28577 & 7.08883 \\
\hline
Bob & -62.06654 & 7.04135 \\
\hline
Eve & -55.98965 & 7.07650 \\
\end{tabular}
\caption{absolute Daten der 4. Messung}
\end{table}

\subsection*{5}
In Abbildung \ref{fig:5} ist eine auf den ersten Blick merkwürdige Entdeckung festzustellen, obwohl Eve direkt neben Bob ist sind die RSSI Werte extrem unterschiedlich. Dies haben wir aber erwartet, da es bereits in der Vorlesung angesprochen wurde. 'Die Korrelation erhöht sich bis zu einer gewissen Distanz, aber wenn man näher ran geht, verringert sie sich sehr stark'. Die Korrelation ist zwar stärker als in 4, aber immernoch sehr sehr inkonsistent. So gibt es mehr Werte in der Nähe von eins aber immernoch extrem viele Ausreißer nach unten.
\begin{figure}[H]
\centering
\subfloat[][]{\includegraphics[width=0.4\textwidth]{Messungen/5_mitBwg_A_B/output_Ex3/correlation_AB.png}} \qquad
\subfloat[][]{\includegraphics[width=0.4\textwidth]{Messungen/5_mitBwg_A_B/output_Ex3/correlation_AE.png}}
\caption{Korrelation mit Bewegung \textbf{Blocksize 10}}
\label{fig:5}
\end{figure}

\begin{table}[H]
\centering
\begin{tabular}{l|l|l}
& Mittelwert & empirische Standardabweichung \\
\hline
Alice & -63.77634 & 7.33027 \\
\hline
Bob & -63.45434 & 7.31302 \\
\hline
Eve & -58.22699 & 7.88637 \\
\end{tabular}
\caption{absolute Daten der 5. Messung}
\end{table}
\subsection*{Fazit}
Abschließend ist zu sagen, dass wir viel Bewegung brauchen um Entropie auf den Channel zu bekommen, um die Standardabweichung zu erhöhen. Des Weiteren ist wichtig, dass der Abstand zwischen Eve und Bob nicht auf intermediate Range kommt, also der perfekte Abstand für eine Angriff ist wohl zwischen 50 cm und 1 cm aber diesem Wert genau zu ermitteln war nicht Teil dieses Versuches. Die Blocklänge glättet die Ergebnisse ab, so fallen einzelne Ausrutscher durch noise nicht mehr auf. \\
Die beste Chance hat der Angreifer generell, wenn wenig Bewegung im Raum ist. So sind die RSSI-Werte homogen und unterscheiden sich nur um einen linearen Wert. Bei Bewegung, egal welchen Abstandes, hat der Angreifer in der Regel keine Chance, da die Korrelation zu schwach ist.\\
Wichtig ist ebenfalls, das der Abstand nicht zu groß wird, da sonst andere physikalische Faktoren überhand gewinnen und die Korrelation zwischen Alice und Bob erheblich sinkt.
\section{Quantisierer Jana Multibit}
\subsection*{a)}
Der Quantisierer nutzt die gesamte Bandbreite der empfangenen Samples. Es werden die Samples empfangen, der kleinste und größte Wert ermittelt, anhand der Differenz die Anzahl der abzuleitenden Bits. Dann wird der Wertebereich in gleich große Intervalle aufgeteilt. Dann wird eine N-Bit Zuordnung für jedes Intervall gesucht. Danach werden die RSS Daten durchgegangen und anhand der Position (welches Intervall) die Secretbits generiert. Es werden pro gemessener RSS Wert mehrere Bits extrahiert.
\subsection*{b)}
\begin{algorithm}[H]
\DontPrintSemicolon
\SetAlgoLined
\KwData{rss\_data []}
range $\gets$ (max $\in$ rss\_data ) - (min $\in$ rss\_data) \;
N $\gets$$\lfloor log_2(range) \rfloor$\;
M $\gets$ $2^N$ \;
splitinterval(range, M)\;
\ForEach{interval}{
	assignToInterval(N\_bit\_value)
}
\ForEach{measurement}{
	\ForEach{interval}{
	\If{rss in interval}{x += extract(interval.getNbit())}
	}
}
stream = reconiliation(x)
\caption{Quantisierer Jana Multibit}
\end{algorithm}
\section{Quantisierer Mathur Suhas}
\subsection*{a)}
Alice und Bob haben einen Vektor an Messdaten, dieser Vektor enthält für jeden Zeitpunkt die berechneten Werte von \textbf{h}. Dann wird ein oberer und untere Threshhold definiert. Alice sucht sich eine Sequenz aus, in der m oder mehr berechnete \textbf{h} Werte über oder unter den Threshholdwerten liegen. Dann sendet Alice Bob den mittleren Zeitpunkt einiger zufällig ausgesuchter Werte, welche zuvor gefunden wurden. Bob schaut nun, ob \textit{m-1} Werte um den Zeitpunkt ebenfalls über oder unter den Thresholds liegen. Bob macht eine Liste, in der alle Matchings stehen und sendet diese Alice. Alice und Bob berechnen nun die Bits von jedem Wert an dem entsprechenden Index. (m muss vorher definiert werden, da steht aber nichts von im Text).
\subsection*{b)}
\begin{algorithm}[H]
\SetKwInOut{Input}{input}
\SetKwInOut{Output}{output}
\DontPrintSemicolon
\SetAlgoLined
\Input{$\hat{h}_a$, $\hat{h}_b$}
\Output{keys $K_a$ = $K_b$}
Alice:\;
\ForEach{element $\in$ $\hat{h}_a$}{
	\If{(element > $q_+$ \textbf{or} element < $q_{-}$) \textbf{and} rand()}{
	store index of element in list L'
	}
}
choose random entrys of L' \& store in L\;
send L to Bob\;
Bob:\;
\ForEach{element $\in$ $\hat{h}_b$}{
	\ForEach{index in L}{
		\If{$\hat{h}_b$(index) > $q_+$ \textbf{or} $\hat{h}_b$(index) < q}{
		store index in $\tilde{L}$
		}
	}	
} 
$K_b$ = Q($\hat{h}_b$($\tilde{L}$))\;
Send $\tilde{L}$ to Alice\;
Alice:\;
$K_a$ = Q($\hat{h}_a$($\tilde{L}$))\;
\caption{Cross-Level Algorithmus}
\end{algorithm}
\section{Bonus: Reading Assignment}
\subsection*{a)}
%(Die Autoren wollen ein mathematisches Modell eines 'Kanal untersuchungs Systems' aufstellen und den Zusammenhang zwischen Biterzeugungsrate und Kanaluntersuchungsrate herstellen. (hört sich auf deutsch ein bisschen blöd an :D). Sie wollen des weiteren anhand der Entropie die Biterzeugungsrate ableiten)\\
The authors want to solve the problem that generating a shared key between two parties from the wireless channel is not possible with dynamically adjustment of the probing rate. This is neccesary to achieve a desired bit generation rate under different moving speeds, different mobile types and different sites. Therefore they have to consider the tradeoff between the bit generation rate, the channel resource consumption and the probing rate according to different scenarios. They made it possible with a mathematical model for channel probing and derive that the bit generation rate ist proportional to the probing rate.
\subsection*{b)}
It is important to use an efficient probing algorithm that adjust the probing rate according to the channel, because if not, the rss sequences would contain many doubled values. In order to avoiding this, the probing algorithm puts up the entropy and increases the efficiency of sampling.
\end{document}